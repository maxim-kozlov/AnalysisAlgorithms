\chapter{Экспериментальный раздел}
\label{cha:research}
    В данном разделе будут проведены эксперименты для проведения 
    сравнительного анализа линейной и конвейерной реализациями системы.

    \section{Сравнительный анализ линейной и конвейерной реализациями системы}
        В рамках данного проекта были проведёны эксперименты по вычислению 
        времени работы системы в линейной и конвейерной реализациях (график \ref{graph:test}).

        Тестирование проводилось на ноутбуке с процессором
        Intel(R) Core(TM) i5-7200U CPU 2.50 GHz \cite{processor-i5-7200u}
        под управлением Windows 10 с 8 Гб оперативной памяти.

        В ходе экспериментов по замеру времени работы было установлено, что 
        конвейерная модель обрабатывает элементы в $ \approx 1.75 $
        быстрее, чем линейная. Это объясняется тем,
        что одновременно обработываются разные элементы на разных этапах.
        
    \begin{figure}[h!]
        \centering
        \begin{tikzpicture}
            \begin{axis}[
                legend pos = north west,
                grid = major,
                xlabel = Количество элементов,
                ylabel = {Время, сек},
                height = 0.5\paperheight, 
                width = 0.75\paperwidth
            ]
            
            \addplot table[x=n, y=linear] {data/test-time.dat};
            \addplot table[x=n, y=conveyor] {data/test-time.dat};
            \legend{
                Линейная,
                Конвейерная
            };
            \end{axis}
        \end{tikzpicture}
        \caption{График зависимости времени работы от модели системы} 
        \label{graph:test}
    \end{figure}

\newpage