\chapter{Экспериментальный раздел}
\label{cha:research}
    В данном разделе будут проведены эксперименты для проведения 
    сравнительного анализа на основе замеров времени работы алгоритмов.

    \section{Сравнительный анализ на основе замеров времени работы алгоритмов}
        В рамках данной работы были проведёны следующие эксперименты по вычислению: 
        \begin{enumerate}
            \item времени работы системы в линейной и конвейерной реализациях (график \ref{graph:test:models});
            \item зависимости работы конвейерной системы от времени работы этапа (график \ref{graph:test:step}).
        \end{enumerate}

        Тестирование проводилось на ноутбуке с процессором
        Intel(R) Core(TM) i5-7200U CPU 2.50 GHz \cite{processor-i5-7200u}
        под управлением Windows 10 с 8 Гб оперативной памяти.

        В ходе экспериментов по замеру времени работы в линейной и конвейерной реализациях было установлено,
        что конвейерная модель обрабатывает элементы в $ \approx 2.6 $ раза
        быстрее, чем линейная. Это объясняется тем,
        что одновременно обработываются разные элементы на разных этапах.
        
        По результатам исследования зависимости времени работы конвейерной системы от времени работы этапа  
        можно сделать вывод, что если одна из стадий намного более трудоемкая, чем остальные,
        то конвейерная обработка становится неэффективной,
        так как производительность всей программы будет упираться в производительность этой стадии,
        и разница между обычной обработкой и конвейерной будет малозаметна.
        В таком случае можно либо разбить трудоемкую стадию на набор менее трудоемких,
        либо выбрать другой алгоритм, либо отказаться от конвейерной обработки.


    \begin{figure}[h!]
        \centering
        \begin{tikzpicture}
            \begin{axis}[
                legend pos = north west,
                grid = major,
                xlabel = Количество элементов,
                ylabel = {Время, мс},
                height = 0.5\paperheight, 
                width = 0.75\paperwidth
            ]
            
            \addplot table[x=n, y=linear] {data/time-models.dat};
            \addplot table[x=n, y=conveyor] {data/time-models.dat};
            \legend{
                Линейная,
                Конвейерная
            };
            \end{axis}
        \end{tikzpicture}
        \caption{График зависимости времени работы от модели системы} 
        \label{graph:test:models}
    \end{figure}

    \begin{figure}[h!]
        \centering
        \begin{tikzpicture}
            \begin{axis}[
                legend pos = north west,
                grid = major,
                xlabel = {Время работы этапа, мкс},
                height = 0.5\paperheight, 
                width = 0.75\paperwidth
            ]
            
            \addplot table[x=t_step, y=t_system] {data/time-step.dat};
            \end{axis}
        \end{tikzpicture}
        \caption{График зависимости отношения времени работы первого этапа к времени работы конвейера} 
        \label{graph:test:step}
    \end{figure}

\newpage