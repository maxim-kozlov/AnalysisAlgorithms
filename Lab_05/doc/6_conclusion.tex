\Conclusion
    В ходе выполнения лабораторной работы 
    были описаны и реализованы алгоритмы конвейерной обработки.
    Разработанный конвейер, с параллельно работающими линиями,
    позволил ускорить работу программы в 2,6 раз
    по сравнению с линейной реализацией. 
    Однако если одна из стадий намного более трудоемкая, чем остальные,
    то конвейерная обработка становится неэффективной,
    так как производительность всей программы будет упираться в производительность этой стадии,
    и разница между обычной обработкой и конвейерной будет малозаметна.
    В таком случае можно либо разбить трудоемкую стадию на набор менее трудоемких,
    либо выбрать другой алгоритм, либо отказаться от конвейерной обработки.
