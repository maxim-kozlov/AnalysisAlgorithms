\chapter{ Аналитический раздел}
\label{cha:analytical}
    цели, задачи мат определения, формулы.

    {\bf Цель}: реализовать и сравнить по эффективности алгоритмы поиска
    расстояния Левенштейна и Дамерау-Левенштейна.


    {\bf Задачи}:
    \begin{enumerate}[1)]
        \item Дать математическое описание расстояния Левенштейна и Дамерау-Левенштейна
        \item Разработать алгоритмы поиска расстояний Левенштейна и Дамерау-Левенштейна
        \item Реализовать алгоритмы поиска расстояний Левенштейна и Дамерау-Левенштейна
        \item Провести эксперименты по замеру времени работы реализованных алгоритмов
        \item Сравнительный анализ реализованных алгоритмов по затраченному времени и максимально затраченной памяти
    \end{enumerate}

    {\bfРасстояние Левенштейна} (или редакционное расстояние) - это минимальное
    количество редакционных операций, которое необходимо для преобразования одной строки в другую.

    Редакционными операциями являются: \begin{enumerate}[1)]
        \item Вставка (I);
        \item Удаление (D);
        \item Замена (R).
    \end{enumerate}

    Расстояние Левенштейна применяется для: \begin{enumerate}
        \item Автозамене
    \end{enumerate}
\newpage