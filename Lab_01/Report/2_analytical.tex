\chapter{ Аналитический раздел}
\label{cha:analytical}
    \section{Цель и задачи практики}

    {\bf Цель}: реализовать и сравнить по эффективности алгоритмы поиска
    расстояния Левенштейна и Дамерау-Левенштейна.

    {\bf Задачи}:
    \begin{enumerate}
        \item дать математическое описание расстояний Левенштейна и Дамерау-Левенштейна;
        \item разработать алгоритмы поиска расстояний Левенштейна и Дамерау-Левенштейна;
        \item реализовать алгоритмы поиска расстояний Левенштейна и Дамерау-Левенштейна;
        \item провести эксперименты по замеру времени работы реализованных алгоритмов;
        \item проанализировать реализованные алгоритмы по затраченному времени и максимально затраченной памяти.
    \end{enumerate}

    \section{Расстояние Левенштейна}

    {\bfРасстояние Левенштейна} (или редакционное расстояние) -- это минимальное
    количество редакционных операций, которое необходимо для преобразования одной строки в другую.

    Редакционными операциями являются: \begin{enumerate}
        \item вставка (I -- Insert);
        \item удаление (D -- Delete);
        \item замена (R -- Replace);
        \item совпадение (M -- Match).
    \end{enumerate}
    
    Операции I, D, R имеют штраф 1, а операция M -- 0.

    Пусть $s_{1}$ и $s_{2}$ — две строки (длиной M и N соответственно) в некотором алфавите V,
    тогда расстояние Левенштейна можно подсчитать по рекуррентной формуле (\ref{formula:Levenshtein}):

    \begin{equation}\label{formula:Levenshtein}
        D(s1[1..i],s2[1..j]) = \left\{ \begin{array}{ll}
            0, & \textrm{$i = 0, j = 0$}\\
            i, & \textrm{$j = 0, i > 0$}\\
            j, & \textrm{$i = 0, j > 0$}\\
           min(s1[1..i],s2[1..j-1]+ 1,\\
           D(s1[1..i-1],s2[1..j]) + 1, &\textrm{$j>0, i>0$}\\
           D(s1[1..i-1],s2[1..j-1]) + M(s1[i], s2[j]),\\
        \end{array} \right.
    \end{equation}
    где $s[1..k]$ - подстрока длиной k и 
    $M(a, b) = \left\{ \begin{array}{ll}
        0, & \textrm{$a = b$}\\
        1, & \textrm{$a \ne b$}\\
    \end{array} \right.$

    В Таблице \ref{table:example:Levenshtein} минимальное расстояние между
    словом "кит" и "скат" равно 2. Последовательность редакторских операций, 
    которая привела к ответу - IMRM.

    \begin{table}[h]
        \caption{Пример работы преобразования слова "кит" в "скат"}
        \centering
        \begin{tabular}{|c|c|c|c|c|c|}
        \hline
  & $\lambda$ & С & К & А & Т \\ \hline
$\lambda$ & 0 & 1 & 2 & 3 & 4 \\ \hline
        К & 1 & 1 & 1 & 2 & 3 \\ \hline
        И & 2 & 2 & 2 & 2 & 3 \\ \hline
        Т & 3 & 3 & 3 & 3 & \cellcolor[HTML]{FFCCC9}2 \\ \hline
        \end{tabular}
        \label{table:example:Levenshtein}
    \end{table}

    \section{Расстояние Дамерау-Левенштейна}  
    Расстояние Дамерау-Левенштейна является модификацией расстояния Левенштейна:
    к операциям вставки, удаления и замены символов, добавлена операция транспозиции (перестановки
    двух соседних символов) (X - exchange). Операция транспозиции возможна, если символы попарно совпадают.

    Пусть $s_{1}$ и $s_{2}$ — две строки (длиной M и N соответственно) в некотором алфавите V,
    тогда расстояние Дамерау-Левенштейна можно подсчитать по рекуррентной формуле (\ref{formula:DamerauLevenshtein}):

    \begin{equation}\label{formula:DamerauLevenshtein}
        D(s1[1..i],s2[1..j]) = \left\{ \begin{array}{ll}
            max(i, j), & \textrm{$min(i, j) = 0$}\\

            min \left\{ \begin{array}{lll}
                s1[1..i], s2[1..j-1] + 1                 \\
                s1[1..i-1], s2[1..j] + 1                 \\
                s1[1..i-2], s2[1..j-2] + 1               \\ 
                s1[1..i-1],s2[1..j-1] + M(s[i], s[j])    \\
                \end{array} \right. & \textrm{$i, j > 1, s1_{i-1} = s2_j, s1_i = s2_{j-1}$}\\

            min \left\{ \begin{array}{lll}
                s1[1..i], s2[1..j-1]+ 1                  \\
                s1[1..i-1], s2[1..j]) + 1                \\ 
                s1[1..i-1],s2[1..j-1] + M(s[i], s[j])    \\
                \end{array} \right. & \textrm{иначе}.\\
        \end{array} \right.
    \end{equation}

\newpage