\chapter{ Аналитический раздел}
\label{cha:analytical}
    {\bf Цель}: реализовать и сравнить по эффективности алгоритмы поиска
    расстояния Левенштейна и Дамерау-Левенштейна.

    {\bf Задачи}:
    \begin{enumerate}[1)]
        \item Дать математическое описание расстояния Левенштейна и Дамерау-Левенштейна
        \item Разработать алгоритмы поиска расстояний Левенштейна и Дамерау-Левенштейна
        \item Реализовать алгоритмы поиска расстояний Левенштейна и Дамерау-Левенштейна
        \item Провести эксперименты по замеру времени работы реализованных алгоритмов
        \item Сравнительный анализ реализованных алгоритмов по затраченному времени и максимально затраченной памяти
    \end{enumerate}

    {\bfРасстояние Левенштейна} (или редакционное расстояние) - это минимальное
    количество редакционных операций, которое необходимо для преобразования одной строки в другую.

    Редакционными операциями являются: \begin{enumerate}[1)]
        \item Вставка (I - Insert);
        \item Удаление (D - Delete);
        \item Замена (R - Replace);
        \item Совпадение (M - Match).
    \end{enumerate}
    Где операции I, D, R имеют штраф 1, а операция M - 0.

    \section{Алгоритм Левенштейна}
    Пусть $s_{1}$ и $s_{2}$ — две строки (длиной M и N соответственно) в некотором алфавите V,
    тогда расстояние Левенштейна можно подсчитать по следующей рекуррентной формуле \ref{formula:Levenshtein}:

    \begin{equation}\label{formula:Levenshtein}
        D(s1[1..i],s2[1..j]) = \left\{ \begin{array}{ll}
            0, & \textrm{$i = 0, j = 0$}\\
            i, & \textrm{$j = 0, i > 0$}\\
            j, & \textrm{$i = 0, j > 0$}\\
           min(s1[1..i],s2[1..j-1]+ 1,\\
           D(s1[1..i-1],s2[1..j]) + 1, &\textrm{$j>0, i>0$}\\
           D(s1[1..i-1],s2[1..j-1]) +  
            \left\{ \begin{array}{ll}
                0, & \textrm{$s1[i] = s2[j]$}\\
                1, & \textrm{$s1[i] \ne s2[j]$}\\
            \end{array} \right.\\
        \end{array} \right.
    \end{equation}
    Где $s[1..k]$ - подстрока длиной k.

    В Таблице \ref{table:example:Levenshtein} минимальное расстояние между
    словом "кит" и "скат" равно 2. Последовательность редакторских операций, 
    которая привела к ответу - IMRM.

    \begin{table}[h]
        \caption{Пример работы преобразования слова "кит" в "скат"}
        \begin{tabular}{|c|c|c|c|c|c|}
        \hline
  & $\lambda$ & С & К & А & Т \\ \hline
$\lambda$ & 0 & 1 & 2 & 3 & 4 \\ \hline
        К & 1 & 1 & 1 & 2 & 3 \\ \hline
        И & 2 & 2 & 2 & 2 & 3 \\ \hline
        Т & 3 & 3 & 3 & 3 & \cellcolor[HTML]{FFCCC9}2 \\ \hline
        \end{tabular}
        \label{table:example:Levenshtein}
    \end{table}

    \section{Алгоритм Дамерау-Левенштейна}
    Расстояние Дамерау-Левенштейна является модификацией расстояния Левенштейна:
    к операциям вставки, удаления и замены символов, добавлена операция перестановки
    двух соседних символов (X - exchange).
    
\newpage