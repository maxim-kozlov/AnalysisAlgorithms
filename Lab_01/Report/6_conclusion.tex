\Conclusion
    В ходе работы были изучены и реализованы алгоритмы нахождения
    расстояния Левенштейна (не рекурсивный с заполнением матрицы,
    рекурсивный без заполнения матрицы, рекурсивный с заполнением матрицы)
    и Дамерау-Левенштейна (не рекурсивный с заполнением матрицы). 
    Выполнено сравнение перечисленных алгоритмов. В ходе экспериментов
    было установлено, что !1! занимают намного больше памяти при 
    обработке длинных строк, чем !2! реализация тех же алгоритмов
    (при длине 1000 символов, !1! использует в n раз меньше памяти,
    чем !2!). Изучены зависимоти времени выполнения алгоритмов от длины
    строк. При анализе времени лучше других себя показал !3!.  
