\Conclusion
    В ходе работы были изучены и реализованы алгоритмы нахождения
    расстояния Левенштейна (не рекурсивный с заполнением матрицы,
    рекурсивный без заполнения матрицы, рекурсивный с заполнением матрицы)
    и Дамерау-Левенштейна (не рекурсивный с заполнением матрицы). 
    Выполнено сравнение перечисленных алгоритмов. В ходе экспериментов
    было установлено, что алгоритмы использующие матрицы занимают намного больше памяти при 
    обработке длинных строк, чем рекурсивная реализация алгоритма.
    Изучены зависимости времени выполнения алгоритмов от длины
    строк. На длинных строках хуже всех себя показала рекурсивная реализация, 
    лучше всех - реализация поиска расстояния Левенштейна не рекурсивно с заполнением матрицы.
    На маленьких строках (до 3 символов) рекурсивная реализация оказалась быстрее и менее затратна по памяти.
