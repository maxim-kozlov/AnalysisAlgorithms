\chapter{ Аналитический раздел}
\label{cha:analytical}
    В данном разделе будут рассмотрены основные теоритические понятия алгоритмов сортировок
    пузырьком с флагом, вставками, выбором.

    \section{Алгоритмы сортировок}
        \subsection{ Алгоритм сортировки пузырьком с флагом}
            Алгоритм сортировки пузырьком или метод простых обменов имеет следующий 
            принцип работы:
            \begin{enumerate}
                \item прохождение по всему массиву;
                \item сравнивание между собой пар соседних ячеек;
                \item если при сравнении оказывается, что значение ячейки i больше, чем значение ячейки i + 1,
                то нужно поменять значения этих ячеек местами.
            \end{enumerate}

            Алгоритм сортировки пузырьком с флагом является модификацией этого алгоритма.
            Идея состоит в том, что если при выполнении прохода методом пузырька не 
            было ни одного обмена элементов массива, то это означает, что массив уже
            отсортирован и остальные проходы не требуются.

        \subsection{ Алгоритм сортировки вставками}
            Алгоритм сортировки вставками, просматривает элементы входной последовательности по одному,
            и для каждого элемента, размещает его в подходящее место среди ранее упорядоченных элементов.
            
            В начальный момент времени отсортированная последовательность пуста. 
            На каждом шаге алгоритма выбирается один из элементов входных данных и
            помещается в нужную позицию в отсортированной последовательности до тех пор,
            пока набор входных данных не будет исчерпан. 
            
            Сортировка методом вставок -- простой алгоритм сортировки. 
            Хотя этот метод сортировки намного менее эффективен,
            чем более сложные алгоритмы (такие как быстрая сортировка), у него есть ряд преимуществ
            \begin{enumerate}
                \item простота реализации;
                \item эффективен на небольших наборах данных;
                \item эффективен на частично отсортированных последовательностях;
                \item является устойчивым алгоритмом (не меняет порядок элементов, которые уже отсортированы).
            \end{enumerate}

            Данный алгоритм можно ускорить при помощи использования бинарного поиска для нахождения места
            вставки текущего элемента в отсортированную часть последовательности.

        \subsection{ Алгоритм сортировки выбором}
            Алгоритм сортировки выбором работает следующим образом: 
            находим наименьший элемент в массиве и обмениваем его с элементом находящимся на первом месте.
            Повторяем процесс -- находим наименьший элемент в последовательности, начиная со второго элемента, и
            обмениваем со вторым элементном и так далее, пока весь массив не будет отсортирован.
            Этот метод называется сортировка выбором, поскольку он работает, 
            циклически выбирая наименьший из оставшихся элементов.

            Главным отличием сортировки выбором от сортировки вставками является, то что в сортировке вставками 
            извлекается из неотсортированной части массива первый элемент (не обязательно минимальный) и
            вставляется на своё место в отсортированной части.
            В отличии от сортировки выбором, где ищется минимальный элемент 
            в неотсортированной части,  который вставляется в конец отсортированной части массива.


    \section{Трудоёмкость алгоритма}
        Трудоёмкость -- количество работы, которую алгоритм затрачивает на обработку данных.
        Является функцией от длины входов алгоритма и позволяет оценить количество работы.

        Введём модель вычисления трудоёмкости.

        \subsection{Базовые операции}
            Ниже представлены базовые операции, стоимость которых единична:
            \begin{enumerate}
                \item $ =, +, +=, -, -=, *, *=,  /, /=, ++, --, \% $,
                \item $ <, \leqslant, ==, \neq, \geqslant , > $,
                \item $ [ $  $ ] $.
            \end{enumerate}
            
        \subsection{Условный оператор}
            if (условие) \{

                // тело A

            \}

            else \{

                // тело B
            
            \}

            Пусть трудоёмкость тела A равна $ f_A $, а тела B $ f_B $, тогда
            стоимость условного оператора можно найти по формуле (\ref{equation:trud:if}):
            \begin{equation}
                f_{if} = f_\text{условия} + \left\{
                    \begin{matrix}
                    min(f_A, f_B) - \text{лучший случай},\\
                    max(f_A, f_B) - \text{худший случай} 
                    \end{matrix}\right.
                \label{equation:trud:if}
            \end{equation}

        \subsection{Цикл со счётчиком}
            for (int i = 0; i < n; i++) \{

                // тело цикла

            \}
            
            Начальная инициализация цикла (int i = 0) выполняется один раз.
            Условие i < n проверяется перед каждой итерацией цикла и при входе в цикл -- n + 1 операций.
            Тело цикла выполняется ровно n раз.
            Счётчик (i++) выполняется на каждой итерации, перед проверкой условия, т.е. n раз.
            Тогда, если трудоёмкость тела цикла равна $ f $, трудоёмкость всего цикла определяется формулой (\ref{equation:trud:for})

            \begin{equation}
                f_\text{цикла} = 2 + n(2 + f)
                \label{equation:trud:for}
            \end{equation}
\newpage