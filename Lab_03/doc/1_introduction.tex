\Introduction    
    Алгоритм сортировки -- это алгоритм для упорядочивания элементов в списке.
    Входом является последовательность из n элементов: $ a_1, a_2, \dots, a_n $
    Результатом работы алгоритма сортировки является перестановка
    исходной последовательности $ a'_1, a'_2, \dots, a'_n $,
    такая что $ a_1 \leqslant a_2 \leqslant \dots \leqslant a_n $, 
    где $ \leqslant $ -- отношение порядка на множестве элементов списка.
    Поля, служащие критерием порядка, называются ключом сортировки.
    На практике в качестве ключа часто выступает число,
    а в остальных полях хранятся какие-либо данные, 
    никак не влияющие на работу алгоритма.
    
    В данной работе рассматриваются три алгоритма:
    \begin{enumerate}
        \item сортировка пузырьком с флагом;
        \item сортировка вставками;
        \item сортировка выбором.
    \end{enumerate}

    Целью данной лабораторной работы является реализация алгоритмов сортировки и
    исследование их трудоемкости.

    Задачи данной лабораторной работы:
    \begin{enumerate}
        \item изучить алгоритмы сортировки пузырьком с флагом, вставками, выбором;
        \item реализовать алгоритмы сортировки пузырьком с флагом, вставками, выбором;
        \item дать оценку трудоёмкости в лучшем, произвольном и худшем случае (для двух алгоритмов сделать вывод трудоёмкости);
        \item провести замеры процессорного времени работы для лучшего, худшего и произвольного случая.
    \end{enumerate}

\newpage