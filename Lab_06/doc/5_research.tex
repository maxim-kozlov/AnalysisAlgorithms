\chapter{Экспериментальный раздел}
\label{cha:research}
    В данном разделе будут проведен  
    сравнительный анализ алгоритмов по затрачиваемому процессорному 
    времени и рассмотрена параметризация муравьиного алгоритма.
    Тестирование проводилось на ноутбуке с процессором
    Intel(R) Core(TM) i5-7200U CPU 2.50 GHz \cite{processor-i5-7200u}
    под управлением Windows 10 с 8 Гб оперативной памяти.

    \section{Сравнительный анализ на основе замеров времени работы алгоритмов}
        Время выполнения алгоритма замерялось с помощью функции Stopwatch \cite{csharp-stopwatch}
        из библиотеки System.Diagnostics. Полученные результаты приведены в таблице \ref{table:test:time}.
        
        \begin{table}[h!]
            \begin{center}
                \begin{tabular}{|c|c|c|}
                    \hline
                    Размер графа & Полный перебор & Муравьиный алгоритм \\ \hline
                    3  & 1 490 & 81 200 \\ \hline
                    5  & 2 500 & 143 100 \\ \hline
                    7  & 43 220 & 698 700 \\ \hline
                    9  & 2 240 905 & 1 290 500 \\ \hline
                    11 & 31 470 700 & 2 100 340 \\
                    \hline
                \end{tabular}
            \end{center}
            \caption{Сравнение времени исполнения алгоритмов решения задачи коммивояжера.}
            \label{table:test:time}
        \end{table}



    \section{Параметризация муравьиного алгоритма}
        Для различных значение параметров

        В таблице \ref{table:test:params} приведена выборка результатов параметризации
        для матрицы смежности размером 10х10. 
        Количество дней принято равным 100. 
        Полным перебором был посчитан оптимальный путь -- он составил 130.

    \begin{table}[h!]
        \caption{Выборка из параметризации для матрицы размером 10 x 10.}
        \label{table:test:params}
        \centering
        \begin{tabular}{|c|c|c|c|c|}
            \hline
            $\alpha$        & $\beta$      & $\rho$      & Длина  & Разница \\
            \hline
            0    & 1    & 0.0    & 130    & 0     \\
            0    & 1    & 0.3    & 130    & 0     \\
            0    & 1    & 0.5    & 131    & 1     \\
            0    & 1    & 1.0    & 130    & 0     \\
            \hline
            0.1  & 0.9  & 0.0    & 130    & 0     \\
            0.1  & 0.9  & 0.3    & 130    & 0     \\
            0.1  & 0.9  & 0.6    & 131    & 1     \\                    
            0.1  & 0.9  & 1.0    & 130    & 0     \\
            \hline
            0.2  & 0.8  & 0.0    & 130    & 0     \\
            0.2  & 0.8  & 0.3    & 131    & 1     \\
            0.2  & 0.8  & 0.6    & 131    & 1     \\
            0.2  & 0.8  & 1.0    & 130    & 0     \\
            \hline
            0.3  & 0.7  & 0.0  & 131    & 1     \\
            0.3  & 0.7  & 0.4  & 130    & 0     \\
            0.3  & 0.7  & 0.9  & 131    & 1     \\
            0.3  & 0.7  & 1.0  & 130    & 0     \\
            \hline
            0.4  & 0.6  & 0.0  & 130    & 0     \\
            0.4  & 0.6  & 0.4  & 131    & 1     \\
            0.4  & 0.6  & 0.5  & 130    & 0     \\
            0.4  & 0.6  & 1.0  & 130    & 0     \\
            \hline
            0.5  & 0.5  & 0.0  & 130    & 0     \\
            0.5  & 0.5  & 0.3  & 131    & 1     \\
            0.5  & 0.5  & 0.7  & 131    & 1     \\
            0.5  & 0.5  & 1.0  & 130    & 0     \\
            \hline
            0.6  & 0.4  & 0.2  & 136    & 6     \\
            0.6  & 0.4  & 0.6  & 133    & 3     \\
            0.6  & 0.4  & 0.7  & 130    & 0     \\
            0.6  & 0.4  & 0.7  & 130    & 0     \\
            \hline
            0.7  & 0.3  & 0.0  & 130    & 0     \\
            0.7  & 0.3  & 0.3  & 134    & 4     \\
            0.7  & 0.3  & 0.6  & 132    & 2     \\
            0.7  & 0.3  & 0.8  & 139    & 9     \\
            \hline
            0.8  & 0.2  & 0.0  & 140    & 10    \\
            0.8  & 0.2  & 0.5  & 134    & 4     \\
            0.8  & 0.2  & 0.7  & 131    & 1     \\
            0.8  & 0.2  & 1.0  & 130    & 0     \\
            \hline
        \end{tabular}
    \end{table}
    
    \begin{table}[h!]
        \caption{Выборка из параметризации для матрицы размером 10 x 10.}
        \label{table:test:params}
        \centering
        \begin{tabular}{|c|c|c|c|c|}
            \hline
            $\alpha$        & $\beta$      & $\rho$      & Длина  & Разница \\
            \hline
            0.9  & 0.1  & 0.0  & 134    & 4     \\
            0.9  & 0.1  & 0.3  & 132    & 2     \\
            0.9  & 0.1  & 0.5  & 134    & 4     \\
            0.9  & 0.1  & 1.0  & 130    & 0     \\
            \hline
            1.0  & 0.0  & 0.0  & 145    & 25     \\
            1.0  & 0.0  & 0.4  & 133    & 3      \\
            1.0  & 0.0  & 0.7  & 142    & 22     \\
            1.0  & 0.0  & 1.0  & 138    & 8      \\
            \hline
        \end{tabular}
    \end{table}

    \section{Вывод}
        В ходе экспериментов по замеру времени работы было установлено, что 
        при небольших размерах графа (от 3 до 7) алгоритм полного перебора 
        выигрывает по времени у муравьиного. 
        Так при размере графа 5, полный перебор работает быстрее примерно в 57 раз.
        Однако при увеличении размера графа (от 9 и выше) муравьиный алгоритм 
        значительно выигрывает по времени у алгоритма полного перебора.
        На размерах графа 11, муравьиный алгоритм работает в 15 раз быстрее.

        Наиболее стабильные результаты автоматической параметризации получаются
        при наборе $\alpha = 0.1..0.5$, $\beta = 0.1..0.5$. 
        При таких параметрах полученный результат не отличается более чем на 1 от эталонного,
        и в около 75\% (на промежутке $\rho = 0.0..1.0$) случаев полученный результат совпадает с эталонным.
        Наиболее нестабильные результаты полученны при $\alpha = 1.0$, $\beta = 0.0$.

\newpage