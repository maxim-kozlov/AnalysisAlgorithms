\chapter{Экспериментальный раздел}
\label{cha:research}
    В данном разделе будут проведен  
    сравнительный анализ алгоритмов по затрачиваемому процессорному 
    времени и рассмотрена параметризация муравьиного алгоритма.
    Тестирование проводилось на ноутбуке с процессором
    Intel(R) Core(TM) i5-7200U CPU 2.50 GHz \cite{processor-i5-7200u}
    под управлением Windows 10 с 8 Гб оперативной памяти.

    \section{Сравнительный анализ на основе замеров времени работы алгоритмов}
        Время выполнения алгоритма в тиках замерялось с помощью функции Stopwatch \cite{csharp-stopwatch}
        из библиотеки System.Diagnostics.
        Матрица смежности заполнялась рандомными числами,
        муравьиный алгоритм инициализировался следующими параметрами:
        maxTime = 10, $\alpha$ = 0.8, $\beta$ = 0.2, $\rho$ = 0.1
        Полученные результаты приведены в таблице \ref{table:test:time}.
        
        \begin{table}[h!]
            \begin{center}
                \begin{tabular}{|c|c|c|}
                    \hline
                    Размер графа & Полный перебор & Муравьиный алгоритм \\ \hline
                    3            & 18             & 461             \\ \hline
                    5            & 164            & 1 398           \\ \hline
                    7            & 1 846          & 4 015           \\ \hline
                    9            & 122 346        & 6 700           \\ \hline
                    11           & 7 600 528      & 8 335           \\ 
                    \hline
                \end{tabular}
            \end{center}
            \caption{Сравнение времени исполнения алгоритмов решения задачи коммивояжера.}
            \label{table:test:time}
        \end{table}      
        
    \section{Параметризация муравьиного алгоритма}
        Для различных значение параметров

        В таблице \ref{table:test:params} приведена выборка результатов параметризации
        для матрицы смежности размером 9х9, 
        заполненная псевдослучайными числами с начальным значение (seed) равным 1.
        Количество дней принято равным 30. 
        Оптимальный путь, полученный полным перебором, составляет 23 у.е.
        В таблице указаны значения параметров, для которых разница с 
        минимальным расстоянием не превышает семи.

    \begin{longtable}[c]{|c|c|c|c|}
        \caption{Выборка из параметризации для матрицы размером 9 x 9.}
        \label{table:test:params} \\

        \hline
        $\alpha$ & $\beta$ & $\rho$ & Разница \\ \hline
        %\endfirsthead
        \endhead
        0.0      & 0.0     & 0.3    & 4       \\ \hline
        0.0      & 0.0     & 0.5    & 6       \\ \hline
        0.0      & 0.0     & 0.8    & 2       \\ \hline
        0.1      & 0.0     & 0.3    & 5       \\ \hline
        0.1      & 0.0     & 0.5    & 4       \\ \hline
        0.1      & 0.0     & 0.8    & 2       \\ \hline
        0.2      & 0.0     & 0.0    & 5       \\ \hline
        0.2      & 0.0     & 0.3    & 5       \\ \hline
        0.2      & 0.0     & 0.5    & 5       \\ \hline
        0.2      & 0.0     & 0.8    & 5       \\ \hline
        0.3      & 0.0     & 0.0    & 2       \\ \hline
        0.3      & 0.0     & 0.3    & 2       \\ \hline
        0.3      & 0.0     & 0.5    & 2       \\ \hline
        0.4      & 0.0     & 0.0    & 3       \\ \hline
        0.4      & 0.0     & 0.3    & 1       \\ \hline
        0.4      & 0.0     & 0.5    & 1       \\ \hline
        0.5      & 0.0     & 0.0    & 0       \\ \hline
        0.5      & 0.0     & 0.3    & 2       \\ \hline
        0.5      & 0.0     & 0.5    & 2       \\ \hline
        0.5      & 0.0     & 0.8    & 1       \\ \hline
        0.6      & 0.0     & 0.0    & 0       \\ \hline
        0.6      & 0.0     & 0.3    & 2       \\ \hline
        0.6      & 0.0     & 0.5    & 3       \\ \hline
        0.6      & 0.0     & 0.8    & 3       \\ \hline
        0.7      & 0.0     & 0.0    & 2       \\ \hline
        0.7      & 0.0     & 0.3    & 1       \\ \hline
        0.7      & 0.0     & 0.5    & 3       \\ \hline
        0.7      & 0.0     & 0.8    & 5       \\ \hline
        0.8      & 0.0     & 0.0    & 0       \\ \hline
        0.8      & 0.0     & 0.3    & 1       \\ \hline
        0.8      & 0.0     & 0.5    & 1       \\ \hline
        0.8      & 0.0     & 0.8    & 4       \\ \hline
        0.9      & 0.0     & 0.0    & 2       \\ \hline
        0.9      & 0.0     & 0.3    & 2       \\ \hline
        0.9      & 0.0     & 0.5    & 2       \\ \hline
        0.9      & 0.0     & 0.8    & 1       \\ \hline
        1.0      & 0.0     & 0.0    & 2       \\ \hline
        1.0      & 0.0     & 0.3    & 0       \\ \hline
        1.0      & 0.0     & 0.5    & 0       \\ \hline
        1.0      & 0.0     & 0.8    & 5       \\ \hline
    \end{longtable}

    \section{Вывод}
        В ходе экспериментов по замеру времени работы было установлено, что 
        при небольших размерах графа (до 9) алгоритм полного перебора 
        выигрывает по времени у муравьиного. 
        Так при размере графа 5, полный перебор работает быстрее примерно в 21 раз.
        Однако при увеличении размера графа (от 9 и выше) муравьиный алгоритм 
        значительно выигрывает по времени у алгоритма полного перебора.
        На размерах графа 11, муравьиный алгоритм работает в 900 раз быстрее.

        Наиболее нестабильные результаты автоматической параметризации получаются
        при наборе $\alpha = 0.1..0.5$, $\beta = 0.1..1$. 
        При таких параметрах полученный результат отличается более, чем на 1 от эталонного.
        Наиболее стабильные результаты полученны при $\alpha = 1.0$, $\beta = 0.0$.

\newpage