\Introduction    
    Умножение матриц -- это одна из самых распространённых опереций над матрицами,
    которая широко применяется в различных численных методах, например, в приложениях 
    для решения системы линейных алгебраических уравений, в программах для 
    преобразований графических структур данных и многих других задачах.

    В данной работе требуется изучить и применить три алгоритма умножения матриц:
    \begin{enumerate}
        \item стандартный алгоритм умножения матриц;
        \item алгоритм Винограда;
        \item оптимизированный алгоритм Винограда.
    \end{enumerate}

    Цель лабораторной работы -- провести сравнительный анализ алгоритмов умножения матриц
    и получить навыки оптимизации трудоёмкости алгоритмов.

    В лабораторной работе ставятся следующие задачи:
     \begin{enumerate}
         \item дать математическое описание формул расчёта умножения матриц для стандарного алгоритма и Винограда;
         \item разработать оптимизированный алгоритм Винограда;
         \item реализовать стандартный алгоритм умножения матриц, Винограда и оптимизированного Винограда;
         \item дать теоритическую оценку трудоёмкости трёх алгоритмов;
         \item провести замеры процессорного времени работы реализаций трёх алгоритмов в худшем и в лучшем случаях.
     \end{enumerate}

\newpage