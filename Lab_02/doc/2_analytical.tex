\chapter{ Аналитический раздел}
\label{cha:analytical}
    В данном разделе будут рассмотрены основные теоритические понятия алгоритмов умножения матриц.

    \section{Алгоритмы умножения матриц}
        \subsection{Классический алгоритм умножения}
    
        Пусть даны две прямоугольные матрицы A размерности MxN и B размерности NxQ
        
        \begin{table}[]
            \centering
            \begin{tabular}{cc}
                $ A = \begin{bmatrix}
                        a_{11} & \cdots  & a_{1n} \\ 
                        \vdots & \ddots  & \vdots \\ 
                        a_{m1} & \cdots  & a_{mn}
                    \end{bmatrix} $, 
                &  
                $ B = \begin{bmatrix}
                        b_{11} & \cdots  & b_{1q} \\ 
                        \vdots & \ddots  & \vdots \\ 
                        b_{n1} & \cdots  & b_{nq}
                    \end{bmatrix} $
            \end{tabular}
        \end{table}

        тогда произведением матриц A и B называется матрица C вида:

        \begin{equation}
            C = AB = \begin{bmatrix}
                        c_{11} & \cdots  & c_{1q} \\ 
                        \vdots & \ddots  & \vdots \\ 
                        c_{m1} & \cdots  & c_{mq}
                    \end{bmatrix},
        \end{equation} 
        где $с_{ij} = \sum _{k=1}^{n} a_{ik} * b_{kj}$

        Классический алгоритм умножения матриц находит матрицу C по определению.

        \subsection{Алгоритм Винограда}
        Шмуэль Виноград предложил алгоритм умножения матриц, в котором используется меньше операций умножения
        в сравнении с классической реализацией, и, следовательно, теоритически быстрее, так как умножение -- долгая операция.

        Рассмотрим два вектора $ U = (u_1, u_2, u_3, u_4) $ и $ V = (v_1, v_2, v_3, v_4)^T $. Их произведение равно 

        \begin{equation}
            UV = u_1v_1 + u_2v_2 + u_3v_3 + u_4v_4
            \label{equation:uv:standart}
        \end{equation}

        Сократим долю умножения среди всех операций -- сгруппируем слагаемые на пары, тогда выражение (\ref{equation:uv:standart}) будет иметь вид

        \begin{equation}
            UV = (u_1 + v_2)(u_2 + v_1) + (u_3 + v_4)(u_4 + v_3) - u_1u_2 - u_3u_4 - v_1v_2 - v_3v_4
            \label{equation:uv:vinograd}
        \end{equation}

        В случаи если длина векторов будет нечётной:
        $ U = (u_1, u_2, u_3) $ и $ V = (v_1, v_2, v_3)^T $,
        выражение (\ref{equation:uv:vinograd}) примет следующий вид (\ref{equation:uv:vinograd:odd}):

        \begin{equation}
            UV = (u_1 + v_2)(u_2 + v_1) - u_1u_2 - v_1v_2 + v_3u_3
            \label{equation:uv:vinograd:odd}
        \end{equation}

        Выражение (\ref{equation:uv:standart}) требует большего количества операций, чем  выражение (\ref{equation:uv:standart}) -- вместо четырёх
        умножений -- шесть, вместо трёх сложений - десять, оно  допускает предварительную обработку.

        Правую часть можно вычислить заранее и хранить для каждой строки первой матрицы 
        и для каждого столбца второй матрицы, что позволяет выполнять для каждого элемента
        лишь два умножения и семь сложений. 
        
        \subsection{Вывод}
            Алгоритм Винограда отличается от классического алгорима умножения матриц меньшим количеством операций умножения, 
            за счёт предварительной обработки строк и столбцов матриц. 

    \section{Трудоёмкость алгоритма}
        Трудоёмкость -- количество работы, которую алгоритм затрачивает на обработку данных.
        Является функцией от длины входов алгоритма и позволяет оценить количество работы.

        Введём модель вычисления трудоёмкости.

        \subsection{Базовые операции}
            Ниже представлены базовые операции, стоимость которых единична:
            \begin{enumerate}
                \item $ =, +, +=, -, -=, *, *=,  /, /=, ++, --, \% $,
                \item $ <, \leqslant, ==, \neq, \geqslant , > $,
                \item $ [ $  $ ] $.
            \end{enumerate}
            
        \subsection{Условный оператор}
            if (условие) \{

                // тело A

            \}

            else \{

                // тело B
            
            \}

            Пусть трудоёмкость тела A равна $ f_A $, а тела B $ f_B $, тогда
            стоимость условного оператора можно найти по формуле (\ref{equation:trud:if}):
            \begin{equation}
                f_{if} = f_\text{условия} + \left\{
                    \begin{matrix}
                    min(f_A, f_B) - \text{лучший случай},\\
                    max(f_A, f_B) - \text{худший случай} 
                    \end{matrix}\right.
                \label{equation:trud:if}
            \end{equation}

        \subsection{Цикл со счётчиком}
            for (int i = 0; i < n; i++) \{

                // тело цикла

            \}
            
            Начальная инициализация цикла (int i = 0) выполняется один раз.
            Условие i < n проверяется перед каждой итерацией цикла и при входе в цикл -- n + 1 операций.
            Тело цикла выполняется ровно n раз.
            Счётчик (i++) выполняется на каждой итерации, перед проверкой условия, т.е. n раз.
            Тогда, если трудоёмкость тела цикла равна $ f $, трудоёмкость всего цикла определяется формулой (\ref{equation:trud:for})

            \begin{equation}
                f_\text{цикла} = 2 + n(2 + f)
                \label{equation:trud:for}
            \end{equation}

\newpage