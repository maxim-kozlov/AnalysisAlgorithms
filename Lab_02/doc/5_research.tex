\chapter{Экспериментальный раздел}
\label{cha:research}
    В данном разделе будут проведены эксперименты для проведения 
    сравнительного анализа трёх алгоритмов по затрачиваемому процессорному 
    времени в зависимости от размеров матриц и чётности / нечётности размеров.

    \section{Сравнительный анализ на основе замеров времени работы алгоритмов}
        В рамках данного проекта были проведёны следующие эксперименты:
        \begin{enumerate}
            \item сравнение времени работы алгоритмов на размерностях матриц 100, 200, 300, 400, 500 (график \ref{graph:test:1})
            \item сравнение времени работы алгоритмов на размерностях матриц 100, 200, 300, 400, 500 (график \ref{graph:test:1})
        \end{enumerate}
        
        Тестирование проводилось на ноутбуке с процессором
        Intel(R) Core(TM) i5-7200U CPU 2.50 GHz \cite{processor-i5-7200u}
        под управлением Windows 10 с 8 Гб оперативной памяти.

    \section{Оценка трудоёмкости алгоритмов умножения матриц}

        \subsection{Стандартный алгоритм}
            Найдём трудоёмкость стандартного алгоритма.
            
            $ f_\text{первый цикл} = 2 + M(2 + f_\text{второй цикл})$  

            $ f_\text{второй цикл} = 2 + Q(2 + f_\text{третий цикл})$  

            $ f_\text{третий цикл} = 2 + N(2 + 11)$  

            $ f_\text{Стандартный} = 13MNQ + 4MQ + 4M + 2 \approx 13MNQ$

        \subsection{Алгоритм Винограда}
            Найдём трудоёмкость алгоритма Винограда.
            
            $ f_\text{первый цикл} = 2 + M(2 + 2 + 3 + \frac{N}{2}(3 + 1 + 6 + 2 + 3)) = \frac{15}{2}MN + 7M + 2$
            
            $ f_\text{второй цикл} = \frac{15}{2}QN + 7Q + 2$

            $ f_\text{третий цикл} = 2 + M(2 + 2 + Q(2 + 7 + 3 + \frac{N}{2}(3 + 1 + 12 + 5 + 5))) = 13MNQ + 12MQ + 4M + 2$

            Условный переход $f_{if} = 2 + \left\{
                \begin{matrix}
                0 - \text{лучший случай},\\
                15QM + 4M + 2 - \text{худший случай} 
                \end{matrix}\right.$

            Итого:
            \begin{equation}
                f_\text{Винограда} = 13MNQ + 12MQ + \frac{15}{2}(MN + QN) + 7(M + Q) + 4M + 8 + 
                    \left\{ \begin{matrix}
                    0 - \text{л.с.},\\
                    15MQ + 4M + 2 - \text{х.с.} 
                    \end{matrix}\right.
            \end{equation}
            $ f_\text{Винограда} \approx 13MNQ $
        \subsection{оптимизированный алгоритм Винограда}

            Найдём трудоёмкость алгоритма Винограда.
                
            $ f_\text{первый цикл}^* = 2 + M(2 + 2 + 2 + \frac{N}{2}(2 + 1 + 5 + 1 + 1)) = 10MN + 6M + 2$
            
            $ f_\text{второй цикл}^* = 10MN + 6M + 2$

            $ f_\text{третий цикл}^* = 2 + M(2 + 2 + Q(2 + 6 + 2 + \frac{N}{2}(2 + 1 + 10 + 4 + 1))) = 9MNQ + 10MQ + 4M + 2$

            Условный переход $f_{if}^* = 2 + \left\{
                \begin{matrix}
                0 - \text{лучший случай},\\
                12QM + 4M + 2 - \text{худший случай} 
                \end{matrix}\right.$

            Итого:
            \begin{equation}
                f_\text{Винограда}^* = 9MNQ + 10MQ + \frac{15}{2}(MN + QN) + 6(M + Q) + 4M + 8 + 
                    \left\{ \begin{matrix}
                    0 - \text{л.с.},\\
                    12MQ + 4M + 2 - \text{х.с.} 
                    \end{matrix}\right.
            \end{equation}
            $ f_\text{Винограда}^* \approx 9MNQ $

    \section{Вывод}
        Несмотря на сложность алгоритма Винограда по сравнению со стандартным, доля умножения в нём меньше.
        Стоит отметить, что алгоритм Винограда имеет худший (матрицы нечётной размерности) и лучший случаи (матрицы чётной размерности),
        в то время как стандартный алгоритм не зависит от чётности размерности матриц.


    \begin{figure}[h!]
        \centering
        \begin{tikzpicture}
            \begin{axis}[
                legend pos = north west,
                grid = major,
                xlabel = Размер матрицы,
                ylabel = {Время, ns},
                height = 0.5\paperheight, 
                width = 0.75\paperwidth
            ]
            
            \addplot table[x=n,y=std] {data/test1.dat};
            \addplot table[x=n,y=vin] {data/test1.dat};
            \addplot table[x=n,y=optVin] {data/test1.dat};
            \legend{
                Стандартный алгоритм,
                Алгоритм Винограда,
                оптимизированный алгоритм Винограда,
            };
            \end{axis}
        \end{tikzpicture}
        \caption{График зависимости времени работы алгоритмов при чётных размерностях матриц} 
        \label{graph:test:1}
    \end{figure}

    \begin{figure}[h!]
        \centering
        \begin{tikzpicture}
            \begin{axis}[
                legend pos = north west,
                grid = major,
                xlabel = Размер матрицы,
                ylabel = {Время, ns},
                height = 0.5\paperheight, 
                width = 0.75\paperwidth
            ]
            
            \addplot table[x=n,y=std] {data/test2.dat};
            \addplot table[x=n,y=vin] {data/test2.dat};
            \addplot table[x=n,y=optVin] {data/test2.dat};
            \legend{
                Стандартный алгоритм,
                Алгоритм Винограда,
                оптимизированный алгоритм Винограда,
            };
            \end{axis}
        \end{tikzpicture}
        \caption{График зависимости времени работы алгоритмов при нечётных размерностях матриц} 
        \label{graph:test:2}
    \end{figure}

\newpage