\chapter{ Аналитический раздел}
\label{cha:analytical}
    В данном разделе будут рассмотрены 
    алгоритмы поиска слов в слове.

    \section{Алгоритм полного перебора}
        Алгоритм полного перебора -- это алгоритм разрешения математических задач,
        который можно отнести к классу способов нахождения решения рассмотрением
        всех возможных вариантов. 
        Полный перебор (или метод «грубой силы», англ. brute force) -- метод решения математических задач.
        Относится к классу методов поиска решения исчерпыванием всевозможных вариантов.
        Сложность полного перебора зависит от количества всех возможных решений задачи.
        Если пространство решений очень велико, то полный перебор может не дать результатов 
        в течение нескольких лет или даже столетий.
        
        В данном случае следует перебирать слова в словаре, 
        пока не встретится нужное слово, 
        следовательно, время работы оценивается как $ O(n)$.

    \section{Алгоритм двоичного поиска}
        Целочисленный двоичный поиск (бинарный поиск) (англ. binary search) -- алгоритм поиска
        объекта по заданному признаку в множестве объектов, упорядоченных по тому же самому признаку,
        работающий за логарифмическое время. 

        Принцип двоичный поиска заключается в том, 
        что на каждом шаге множество объектов делится на две части 
        и в работе остаётся та часть множества, 
        где может находится искомый объект. 
        В зависимости от постановки задачи,
        процесс может остановливается, 
        когда получен первый или же последний индекс вхождения элемента.
        Последнее условие -- это левосторонний/правосторонний двоичный поиск. 

    \section{Алгоритм поиска по сегментам}
        Суть данного алгоритма заключается в том,
        что необходимо разбить словарь на сегменты.
        Каждый сегмент определяет первую букву слов,
        которые находятся в нем. 
        Для того, чтобы найти слово в таком
        словаре необходимо определить сегмент, где может находиться слово,
        а после произвести поиск в данном сегменте.

\newpage