\Introduction
    Словарь -- книга или любой другой источник,
    информация в котором упорядочена c помощью разбивки на небольшие статьи,
    отсортированные по названию или тематике. 
    Различают энциклопедические и лингвистические словари.
    С развитием компьютерной техники всё большее распространение получают электронные словари и онлайн-словари.
    Первым русским словарём принято считать Азбуковник,
    помещённый в списке Кормчей книги 1282 года и содержащий 174 слова.
    Задача состоит в поиске слов из словаря в случайных данных любого размера(напр. в файле).
    Поскольку словарь меняется редко, то можно его подготовить
    (напр. отсортировать, создать дерево итд). 
    Это зависит от алгоритма поиска, который будет использован. 

    Целью данной лабораторной работы является реализация 
    алгоритмов поиска слов в словаре и исследование их трудоемкости.

    Задачи данной лабораторной работы:
    \begin{enumerate}
        \item описать алгоритм полного перебора;
        \item описать алгоритм двоичного поиска;
        \item описать алгоритм поиска слов по сегментам;
        \item реализовать 3 алгоритма поиска по словарю;
        \item провести замеры времени работы алгоритмов.
    \end{enumerate}

\newpage